\documentclass[12pt]{article}
\usepackage{amsmath, amssymb, amsthm, booktabs}
\usepackage{geometry}
\geometry{a4paper, margin=1in}
\usepackage{hyperref}

\title{Summary of Key Probability Distributions and Statistical Formulas}
\author{}
\date{\today}

\begin{document}
\maketitle

\tableofcontents
\newpage

\section{General Probability and Statistics Formulas}

\subsection{Conditional Probability}
The probability of an event $A$ given another event $B$ has occurred is given by:
\begin{align*}
    P(A \mid B) = \frac{P(A \cap B)}{P(B)}, \quad \text{if } P(B) > 0.
\end{align*}
\paragraph{Example:} Suppose a fair six-sided die is rolled. The probability of rolling an even number given that the number is greater than 2 is:
\begin{align*}
    P(E \mid G) = \frac{P(E \cap G)}{P(G)} = \frac{\frac{2}{6}}{\frac{4}{6}} = \frac{1}{2}.
\end{align*}

\subsection{Central Limit Theorem (CLT)}
The CLT states that for a sequence of i.i.d. random variables $X_1, X_2, ..., X_n$ with mean $\mu$ and variance $\sigma^2$, their normalized sum converges in distribution to a normal distribution:
\begin{align*}
    \frac{\sum_{i=1}^{n} X_i - n\mu}{\sigma \sqrt{n}} \to N(0,1) \quad \text{as } n \to \infty.
\end{align*}

\subsubsection{When to Use CLT}
\begin{itemize}
    \item \textbf{For Normal Distributions:} If $X_i \sim N(\mu, \sigma^2)$, then the sample mean $\bar{X}$ also follows a normal distribution exactly for any $n$. The CLT is unnecessary in this case.
    \item \textbf{For Other Distributions:} If $X_i$ is not normally distributed but has finite variance, the CLT ensures that the sum (or mean) of a sufficiently large number of independent samples is approximately normally distributed.
    \item \textbf{For Poisson Distributions:} If $X_i \sim \text{Poisson}(\lambda)$, then for large $n$, the sum $S_n = X_1 + X_2 + ... + X_n$ is approximately normal:
    \begin{align*}
        S_n \approx N(n\lambda, n\lambda).
    \end{align*}
\end{itemize}

\paragraph{Example:} Suppose we roll a fair die 1000 times and compute the mean outcome. Each roll has expected value $E[X] = 3.5$ and variance $\operatorname{Var}(X) = \frac{35}{12}$. By the CLT,
\begin{align*}
    \bar{X} \approx N\left(3.5, \frac{35}{12000}\right).
\end{align*}
Thus, for large $n$, we can approximate probabilities of sample means using the normal distribution.

\section{Conclusion}
This document summarizes key probability distributions, expected value and variance properties, covariance and correlation, and fundamental statistical estimation techniques. These principles are crucial in probability theory and data analysis.

\end{document}
